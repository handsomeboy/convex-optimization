\documentclass{paper}

%\usepackage{times}
\usepackage{epsfig}
\usepackage{graphicx}
\usepackage{amsmath}
\usepackage{amssymb}
\usepackage{color}


% load package with ``framed'' and ``numbered'' option.
%\usepackage[framed,numbered,autolinebreaks,useliterate]{mcode}

% something NOT relevant to the usage of the package.
\setlength{\parindent}{0pt}
\setlength{\parskip}{18pt}






\usepackage[latin1]{inputenc} 
\usepackage[T1]{fontenc} 

\usepackage{listings} 
\lstset{% 
   language=Matlab, 
   basicstyle=\small\ttfamily, 
} 



\title{Image inpainting -- Assignment 2}



\author{Mich\`ele Wyss \\10-104-123}
% //////////////////////////////////////////////////


\begin{document}



\maketitle


% Add figures:
%\begin{figure}[t]
%%\begin{center}
%\quad\quad   \includegraphics[width=1\linewidth]{ass2}
%%\end{center}
%
%\label{fig:performance}
%\end{figure}

\section{Primal-dual formulation}
Our objective is to find the image $\tilde u$ that satisfies 
$$\tilde u = \arg \min_u \frac{\lambda}{2} \|u-g\|_\Omega^2 + \|\nabla u\|_2.$$

\noindent We can reformulate the objective in a more abstract way as follows:
$$\min_{u}F(Ku) + G(x)$$
where $K = \nabla$, $F(u) = \|u\|_2$, $G(u) = \frac{\lambda}{2}\|u-g\|_\Omega^2$.

\noindent The primal-dual formulation of this problem is then given by:
$$\min_u \max_p \langle p,\nabla u \rangle - F^*(x) + \frac{\lambda}{2} \|u-g\|_\Omega^2,$$
Where the Legendre-Fenchel transform of $F$ can easily be found as follows:
\begin{align*}
 F^*(y) = (\|\cdot \|_2)^* (y) &= \sup_x x^Ty - \|x\|_2 \\
 &= \sup_x x^Ty - \max_{\|z\|_2 \leq 1} x^Tz \\
 &= \sup_x \min_{\|z\| \leq 1} x^T (y-z) \\
 &= \begin{cases} 0 ~ \text{if } \|y\| \leq 1, \\ \infty ~ \text{otherwise.} \end{cases}
\end{align*}


\section{To do's}
\begin{enumerate}
\item Write the Primal-Dual formulation for this problem..

\item Write the explicit expressions for the Primal-Dual steps $y^{n+1} = \text{prox}_{\sigma F^*} (y^n + \sigma K \bar{x}^n)$ and $x^{n+1} = \text{prox}_{\tau G} (x^n  - \tau K^* y^{n+1})$.


\item \textbf{Implement primal-dual method for inpainting.} In this section you should:

\begin{itemize}
\item Show some images, as the the primal-dual method progresses iteration by iteration. Display the initial and the final image and 3 more images in between.
\end{itemize}
\item \textbf{ Find optimal $\lambda$.} In this section you should:

\begin{itemize}
\item Display the $SSD$ vs. $\lambda$ graph.
\item Describe the effect of $\lambda$ with respect to the $SSD$ between the ground truth and the solution image.
\end{itemize}

\item \textbf{ Conclusions.} Discuss the two methods. In this section you should:
\begin{itemize}
\item Discuss the advantages and disadvantages of each method.
\end{itemize}
\end{enumerate}

 \end{document}
 
 